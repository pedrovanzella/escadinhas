\documentclass[12pt]{article}
\usepackage[T1]{fontenc}
\usepackage[latin9]{inputenc}
\usepackage{algorithm2e}

\makeatletter
%%%%%%%%%%%%%%%%%%%%%%%%%%%%%% Textclass specific LaTeX commands.
\newenvironment{lyxlist}[1]
{\begin{list}{}
{\settowidth{\labelwidth}{#1}
 \setlength{\leftmargin}{\labelwidth}
 \addtolength{\leftmargin}{\labelsep}
 \renewcommand{\makelabel}[1]{##1\hfil}}}
{\end{list}}

%%%%%%%%%%%%%%%%%%%%%%%%%%%%%% User specified LaTeX commands.
\usepackage{sbc-template}

\usepackage[brazil]{babel}

\sloppy

\title{Números Escadinha}

\author{Pedro Vanzella\inst{1}}

\address{Faculdade de Informática -- Pontifícia Universidade Católica do Rio
Grande do Sul (PUCRS) \\ Av. Ipiranga, 6681 - Porto Alegre / RS / Brasil
    \email pedro@pedrovanzella.com}

\makeatother

\usepackage{babel}
\usepackage{listings}
\lstset {
    mathescape,
    frame=none
}
\renewcommand{\lstlistingname}{Listagem}

\begin{document}

\maketitle
\begin{abstract}
A solution for the problem of finding the number of Stairway Numbers
in an arbitrary base-n is proposed.
\end{abstract}
\begin{resumo}
Propõe-se uma solução para o problema de se encontrar a quantidade de Números Escadinha em uma base arbitrária n.
\end{resumo}


\section{Introdução}\label{section:intro}

São chamados Números Escadinha aqueles que seguem um conjunto simples
de regras:
\begin{lyxlist}{00.00.0000}
\item [{1.}] Não começam com o dígito zero
\item [{2.}] Não possuem dígitos repetidos
\item [{3.}] Entre um dígito e o seguinte, a diferença em módulo não é
superior a 2.
\end{lyxlist}
O problema a ser resolvido neste artigo é o de se encontrar quantos
Números Escadinha existem em uma dada base. Fica evidente que eles
são finitos, devido à Regra 2 - o número de dígitos é limitado ao
tamanho da base, senão dígitos começariam a se repetir.


\section{Primeira Tentativa}\label{section:primeira}

Uma tentativa inicial foi feita, tentando o algorítmo mais óbvio e
ingênuo.

Para evitar conversões de base, foi usado um array de inteiros, onde
cada elemento do array representa um dígito do número. O problema
desta parte da solução é que incrementar um dígito roda, na pior das
hipóteses, em tempo linear, oposto a incrementar um inteiro simples,
que deve rodar em tempo constante. Isto é, para cada número a se incrementar,
é necessário verificar o overflow no último dígito. Caso isto ocorra,
faz-se necessário verificar por overflow em cada dígito anterior,
até que não haja mais dígitos em overflow, ou que se tenha chegado
ao começo do array. Este algoritmo foi utilizado em soluções posteriores
pela sua praticidade. 

\subsection{Algorítmos}\label{section:primeira:algoritmos}

O algorítmo de busca por números escadinha faz algumas chamadas a outros algorítmos que valem a pena ser mencionados. Vejamos cada um deles.

\subsubsection{Escadinha}\label{section:primeira:algoritmos:escadinha}
O algoritmo de procura por números válidos é o mais ingênuo possível:
itera por todos os números possíveis na base e, um por um, verifica
se ele se adequa às regras.

\begin{lstlisting}
escadinha(base):
  count $\leftarrow$ 0
  num $\leftarrow$ [0] 
	
  enquanto num $\neq$ maxnum: 
    num $\leftarrow$ incrementa(num, base) 
    se valido(num):
      count++

  $\rightarrow$ count
\end{lstlisting}


No algorítmo acima, vemos um {\sf maxnum}.
Ele pode ser facilmente computado gerando um array de base-posições
com $base-1$ em cada posição. 

\subsubsection{Checagem de Validade}\label{section:primeira:algoritmos:valido}
Ainda no algoritmo escadinha, vemos uma chamada para {\sf valido(num, base)}.
Vamos analizar esta chamada, pois nela se encontra o maior problema
de performance deste algorítmo.

\begin{lstlisting}
valido(num):
  para i de 0 a len(num):
    se $|num[i] - num[i - 1]| > 2$:
      $\rightarrow$ False

  $\rightarrow$ unique(num)
\end{lstlisting}


Ignorando {\sf unique(num)}, vemos que este algoritmo roda em tempo linear
em relação à quantidade de dígitos de {\sf num}. O problema mesmo, como
veremos, está na verificação de unicidade dos dígitos de um número.
Vejamos o algoritmo que faz isto:

\begin{lstlisting}
unique(num):
  para i de 0 a len(num):
    para j de 0 a len(num):
      se i $=$ j:
        continue 
      se num[i] $=$ num[j]:
        $\rightarrow$ False 
  $\rightarrow$ True 
\end{lstlisting}


Claramente, este algoritmo roda em $O(base^{2})$, pois para
cada dígito de {\sf num}\footnote{Cujo tamanho é igual a {\sf base}}, é necessário verificar sua igualdade contra todos
os demais dígitos. Este é um candidato óbvio para otimização. Seria
ideal se, em uma tentativa futura, pudéssemos eliminar completamente
a chamada a este método, tendo a confiança de que só iremos visitar
números cujos dígitos são únicos entre si. No entanto, veremos que apenas otimizar esta parte do algorítmo não reduz a ordem do problema.

\subsubsection{Análise de Complexidade}\label{section:primeira:complexidade}
De fato, vemos que o algoritmo escadinha, considerando tudo, tem uma
complexidade $O(base^{base})$, pois o maior número a ser testado é um de {\sf base}-dígitos-{\sf base-1}, como, por exemplo, $44444$ para base-5.
% TODO: uma explicação melhor %

\subsection{Resultados}\label{section:primeira:resultados}
Obviamente o algoritmo é ruim. Mas quão ruim? Seria possível resolver o problema para uma entrada razoavel? A fim de curiosidade, o algorítmo foi implementado e os resultados podem ser vistos na Tabela~\ref{table:resultados-1}.

\begin{table}[h]
\caption{Primeira Tentativa}
\label{table:resultados-1}
\begin{tabular}{ll}
  {\sf num} & Tempo (s) \\
  \hline
  2 & 0.01 \\
  3 & 0.01 \\
  4 & 0.01 \\
  5 & 0.02 \\
  6 & 0.09 \\
  7 & 1.25 \\
  8 & 23.9 \\
  9 & 540.9    
\end{tabular}
\end{table}

Para entradas menores a 4, a diferença de tempo de execução é imperceptível. Para entradas até 7, vemos um crescimento grande do tempo de execução, mas ainda nada terrível. Para 8, a diferença de tempo já é gritante. Em base-9, o tempo total de execução foi para quase 10 minutos. Esta diferença só aumenta, de maneira assustadora, a cada execução com base mais alta. Fica claro que para base-11 será impossível executar o algorítmo em tempo hábil.

Uma solução mais inteligente é necessária.

\section{Segunda Tentativa}\label{section:segunda}

Como vimos na Seção~\ref{section:primeira:complexidade}, o primeiro algorítmo cresce na ordem de $O(base^{base})$.

Mas a Regra 2 nos diz que não podem haver dígitos repetidos, então este maior número já pode ser descartado. Até onde podemos descartar números, então? Simples: O primeiro dígito será o maior número da base. Os dígitos seguintes serão o dígito anterior, menos um. Este é, claramente, o maior número de uma base.

Infelizmente, como podemos ver na Seção~\ref{section:segunda:resultados}, isto não altera a performance do algorítmo, mesmo reduzindo significativamente a quantidade de números a serem testados. A complexidade do algoritmo continua $O(base^{base})$.

\subsection{Resultados}\label{section:segunda:resultados}

Os resultados da Tabela~\ref{table:resultados-2} são essencialmente iguais aos da Tabela~\ref{table:resultados-1}. Isto é esperado, já que a complexidade do algorítmo não foi alterada.

\begin{table}[h]
\caption{Segunda Tentativa}
\label{table:resultados-2}
\begin{tabular}{ll}
  {\sf num} & Tempo (s) \\
  \hline
  2 & 0.01 \\
  3 & 0.01 \\
  4 & 0.01 \\
  5 & 0.02 \\
  6 & 0.08 \\
  7 & 1.21 \\
  8 & 23.9 \\
  9 & 535.2
  \end{tabular}
  \end{table}

Precisamos de uma solução mais inteligente ainda.

\section{Terceira Tentativa}\label{section:terceira}
Vale a pena pararmos um pouco para analisar como os números são formados. A Seção~\ref{section:primeira} nos dá uma pista da maneira com que isto pode ocorrer. Podemos gerar outro número de uma base de duas maneiras: incrementando o último dígito, ou adicionando mais um dígito ao final do mesmo. Vamos analisar a segunda opção.

Digamos que tenhamos um número inválido para o problema. Fica fácil de ver que não adianta testar nenhum número que comece com os mesmos dígitos que ele, pois todos serão inválidos, pelo mesmo motivo que o primeiro já era. Isso nos deixa somente com a tarefa de testar números que variem o último dígito em relação a ele.

Pensando desta maneira, podemos ver o problema como uma árvore, onde cada galho é um dígito a mais sendo inserido ao final do número já existente. Assim que encontramos um galho que gera um número inválido para o problema, podemos desistir daquele ramo inteiro, pois todos os galhos derivados dele também serão inválidos.

Isto é um algorítmo de {\em Branch and Bound}, e a melhor maneira de implementá-lo é recursivamente.

\subsection{Algoritmo}\label{section:terceira:algoritmo}
Veja que o algorítmo é bastante conciso:

\begin{lstlisting}
escadinha(base, num, d, usad):
  count $\leftarrow$ 0
  se len(num) $\geq$ base:
    return 0
  se num $=$ [] ou ($|num[-1] - d| \leq 2$ and not usad[d]):
    num.append(d)
    usad[d] = True
    count += 1
    para i de 0 a base - 1:
      count += escadinha(base, num, i, usad)
  return count
\end{lstlisting}

A chamada ao algorítmo deve ser algo como:
\begin{lstlisting}
count $\leftarrow$ 0
para i de 1 a base - 1:
  count += escadinha(base, [], i, [False for x de 0 a base-1], 0)
\end{lstlisting}

Algumas observações a respeito da notação são interessantes: {\sf []} representa um array vazio. {\sf num.append(d)} insere {\sf d} no final de {\sf num}.

Há três pontos de retorno no algorítmo recursivo\footnote{Claramente, os pontos de retorno poderiam ser reduzidos para dois, mas foram deixados assim por simplicidade de leitura}: Se estamos tentando inserir um dígito a mais que a base permite, retornamos zero. Se o dígito a ser inserido não satisfaz as condições do problema, também retornamos zero. Se o dígito satisfaz as condições do problema, retornamos $1$, mais a contagem de números válidos da subárvore deste dígito.

O {\em array} de booleanos {\sf usad} é utilizado para evitar termos de pesquisar no número cada vez que quisermos saber se um dígito já foi utilizado neste número ou não. Uma pesquisa destas rodaria em $O(n)$\footnote{Itera-se pelo número, dígito a dígito. Ao encontrar um número, retorna-se Falso. Somente temos certeza de um número estar disponível após percorrer todos os {\em n} dígitos do número.}, pois seria necessário percorrer o número, testando dígito a dígito. Em vez disso, cada vez que usamos um dígito, marcamos um booleano como verdadeiro no {\em array}. Deste modo, a verificação ocorre em $O(1)$, ao custo de um pouco de memória.

\subsection{Análise de Complexidade}\label{section:terceira:complexidade}

Analisar a complexidade de um algorítmo recursivo é um pouco mais complicado. Podemos ver, de maneira empírica, pela Tabela~\ref{table:resultados-3}, que o algorítmo é muito melhor que o apresentado na Seção~\ref{section:segunda} e na Seção~\ref{section:primeira}. Isto já é um bom sinal, ainda que não prove nada, formalmente.

Se virmos a execução do algorítmo como uma árvore\footnote{O que é uma representação bem realista}, podemos contar quantas execuções seriam feitas, caso todos os números fossem válidos. Este é a quantidade de números possíveis de serem gerados por uma base, e isto é trivial de ser calculado: $base^{base}$. A complexidade deste algorítmo deve ser, portanto, $O(base^{base})$. Ora, mas esta é a mesma complexidade dos algorítmos apresentados nas Seções~\ref{section:primeira} e~\ref{section:segunda}!

De fato, se visitássemos todos os números possíveis, seria esta a complexidade. No entanto, assim que encontramos um número impossível de acordo com as regras, desistimos de toda a sub-árvore. No fim das contas, isto nos poupa um esforço enorme, reduzindo enormemente o custo prático do algorítmo.

\subsection{Resultados}\label{section:terceira:resultados}

Como podemos ver na Tabela~\ref{table:resultados-3}, o algorítmo resolve o problema em um tempo muito curto. De fato, ele resolve para uma entrada de 14 em menos tempo que o algorítmo da Seção~\ref{section:segunda} resolve para uma entrada de 7.

Mais importante que isto, o seu custo cresce de maneira muito menos agressiva do que o algorítmo anterior, sendo possível resolver para bases muito mais altas antes que a impaciência nos vença.

\begin{table}[h]
\caption{Terceira Tentativa}
\label{table:resultados-3}
\begin{tabular}{ll}
  {\sf num} & Tempo (s) \\
  \hline
  2 & 0.02 \\
  3 & 0.02 \\
  4 & 0.02 \\
  5 & 0.02 \\
  6 & 0.02 \\
  7 & 0.02 \\
  8 & 0.02 \\
  9 & 0.03 \\
 10 & 0.05 \\
 11 & 0.09 \\
 12 & 0.16 \\
 13 & 0.31 \\
 14 & 0.63 \\
 15 & 1.21 \\
 16 & 2.33 \\
 17 & 4.76 \\
 18 & 8.95
  \end{tabular}
  \end{table}
  
Este algorítmo é tão mais rápido que foi necessário rodá-lo com entradas superiores à proposta para termos uma idéia empírica de como seu custo cresce de acordo com a entrada.

Por sinal, a resposta para base 11 é $5748$ números.


\end{document}
